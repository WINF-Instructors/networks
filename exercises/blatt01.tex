%!TEX TS-program = pdflatex
%!TEX TS-options = -shell-escape
% % % % %   Die folgenden Zeilen müssen ihre Zeilennummern 4 und 5 behalten !!!    % % % % %
\newcommand{\printpraesenzlsg}{false}
\newcommand{\printloesungen}{false}
\newcommand{\printbewertungen}{false}
% % % % %   \newcommand{\printloesungen}{false}                                    % % % % %
\newcommand{\blattnummer}{1}
%\newcommand{\abgabetermin}{\textcolor{red}{bis 11.04.2022, 09:00 Uhr}}
\input{include/config.tex}

% Änderungen 2020: Hinweise auf Digitallehre angepasst; Rechnergeschichte entfernt; Blatt 1 und 2 zusammengefasst.
\begin{document}
\iforiginal{\input{include/kopf.tex}}

% \begin{notes} \small
% 	\textbf{Abgabetermine für Blatt 1:}
	
% 	Aufgaben 1.2/1.3: Montag, 11. April, 09:00 Uhr \\
% 	Aufgabe 1.4: Mittwoch, 20. April, 18:00 Uhr
% \end{notes}

\aufgabetitel{$12$}{Apache Webserver} \\
Ziel der Aufgabe ist es, einen Apache Webserver zu installieren.
Wir tun das auf drei Arten (s.u.). Der Webserver soll installiert werden, die Website angezeigt, die Website modifiziert und das Update ebenfalls angezeigt werden.

Die Varianten der Installation sind
\begin{enumerate}[(i)]
  \item auf einer virtuellen Maschine (ubuntu server auf Hyper-V oder Virtualbox)
  \item in einem Docker-Container
  \item auf einer AWS EC2-Instanz
\end{enumerate}

\aufgabetitel{$5$}{Python API}\\
Schreibt eine kleine API in Python (flask), die eine Ressource "uber einen Pfad exponiert (z.B. einfach "/") und ein einfaches GET erm"oglicht (also einen Text oder ein html file zur"uckliefert, der dann angezeigt wird.)
Probiert aus, ob das funktioniert (Docker oder lokal oder auf einem Server)

\aufgabetitel{$5$}{Chat "uber Sockets}
Checkt den Beitrag \texttt{https://www.geeksforgeeks.org/simple-chat-room-using-python/} und vollzieht den Code nach. Erkl"art den Code bei der n"achsten Session.
\end{document}
